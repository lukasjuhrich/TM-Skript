\section{kleine Schwingungen}
%
$\rightarrow$ Resonantphänomene:\\
\emph{Resonanz:}bei einer bestimmten Frequenz schwingt ein gekoppeltes Vielteilchensystem  besonders stark.
\emph{Beispiele:}
\begin{enumerate}
	\item mechanische konstruktionen (Fahrzeugbau) sollten keine Resonanzen aufweisen ($\rightarrow$ Hubschrauber-Boden-Resonanz\footnote{\url{https://www.youtube.com/watch?v=bs2rNBJ6D3A}})
	\item Brücke
	\item Wolf (Streichinstrumente)
\end{enumerate}
\emph{Problem:} Es gibt kollektive Schwingungen einer Frequenz bei kopplung einzelner schwingungsfähiger Freiheitsgrade
\begin{itemize}
	\item 'Eigenfrequenzen' des gekoppelten Systems
	\item Eigenmoden -- " --
\end{itemize}
%
%
\skizze{schwingungen gleich und gegenphasig}
%
\subsection{Lineare Differenzialgleichungen (2.Ordnung)}
%
\subsubsection{Beispiel}
%
\begin{flalign}
	\ddot{x}+a\dot{x}+bx=f(t)
\end{flalign}
\begin{description}
	\item[linear]~\par $x$ tritt nur linear auf
	\desc{2.Ordnung} $\ddot{x}=\diff{^^2}{t^2}x$ zweite Ableitung
	\desc{homogen} $f(t)=0$
	\desc{inhomogen} $f(t)\neq 0$
	\desc{wichtig} für lineare, homogene Differentialgleichungen gilt ein Superpositionsprinzip mit $x_1(t), x_2(t)$ auch $\alpha x_1(t)+\beta x_2(t)$ für beliebige $\alpha ,\beta\in\M{R}$ eine Lösung der Differentialgleichung
\end{description}
