\section{Lagrange-Formalismus}
\subsection{Lagrange I}
\subsection{Lagrange II}
Betrachte folgend ein System mit $f$ Freiheitsgraden.
\begin{itemize}
\item $(q_1,\cdots,q_f)$ – „Generalisierte
  real.(?) Koordinaten“
\item ${q_L}$, $L=1,\cdots,f$ generalisierte Koordinate (?)
\item $(\dot{q_1}\cdots,\dot{q_f})$ „Verallgemeinerte
  Geschwindigkeiten“
\item Lagrange-Funktion: $L := T - U$\footnote{Anm.d.Skr.: im
    Englischen Wird die Potentielle Energie häufig mit $V$ bezeichnet,
    weswegen man oft (vor Allem auf Wikipedia) auf $L=T-V$ trifft.}
\item
  $L = L(\underbrace{q_1, \cdots, q_f}_f, \underbrace{\dot{q_1},
    \cdots, \dot{q_f}}_{f}, t) = L(q_L, \dot{q_L}, t) = L(q, \dot{q},
  t)$
% TODO markieren
\item Euler-Lagrange-Gleichungen:
  \begin{align*} \frac{d}{dt}\frac{\partial L}{\partial
      \dot{q_{\mathcal{L}}}} \frac{\partial L}{\partial
      q_{\mathcal{L}}} &= 0 & \mathcal{L} \in \left\{ 1,\cdots,f
    \right\}
  \end{align*}
\item Führt zu $f$ gekoppelten DGLs 2. Ordnung
\end{itemize}

\paragraph*{Beispiele:}
\subparagraph{Teilchen in 3D unter Einfluss eines Ortsabhängigen  Potentials}
\begin{itemize}
\item \emph{Keine Zwangsbedingungen} $\Rightarrow$
  $T = \frac{1}{2}m\dot\vec{r}$, $U=U \left( \vec{r} \right)$
\item $f=3$
\item Wir wählen \emph{kartesische Koordinaten}: $\left( q_1,q_2,q_3 \right) = (x,y,z)$
\item $\Rightarrow$
  $T = \frac{1}{2}m \left( \dot{x}^2 + \dot{y}^2 + \dot{z}^2 \right)$,
  $U = U(x,y,z)$ mit
  $\vec{r} = \vec{r}(q) = (x,y,z)^T$
\item $\Rightarrow$ für $\mathcal{L}\in \left\{ 1,2,3 \right\}$
  $L = L(x,y,z,\dot{x},\dot{y},\dot{z},t) = \frac{1}{2}m \left(
    \dot{x}^2 + \dot{y}^2 + \dot{z}^2 \right) - U(x,y,z)$
\item Euler-Lagrange-Gleichungen: Sei $\mathcal{L}=1$
% TODO fix alignment
\begin{align*}
  \Leftrightarrow
  \frac{d}{dt}\frac{\partial L}{\partial \dot{x}} \frac{\partial L}{\partial x} = 0\\
  \frac{\partial L}{\partial \dot{x}} = m\dot{x}
  \frac{\partial L}{\partial x}
&= - \frac{\partial U}{\partial x}\\
  \frac{d}{dt}m\dot{x} + \frac{\partial U}{\partial x}
&= 0
  \Leftrightarrow m \ddot{x} = - \frac{\partial U}{\partial x}
\end{align*}
\item Entsprechender Ansatz für $y,z$ führt insgesamt zu $m\ddot\vec{r}=-\nabla U$
\item Wie erwartet: Newton-Ansatz bestätigt!
\end{itemize}

\subparagraph{Mit Zwangsbedingungen: Mehre Beispiele}
\begin{enumerate}

\item\label{item:8} Rolle

\skizze{skizze}

\begin{itemize}
\item Hier: $f=1$, da (1D+1D)-Bewegung gekoppelt
\item Wähle Höhe $q_1=h$ (alternativ: Winkel $\phi$) als
  generalisierte Koordinate
\item $L = L (h,\dot{h}) = T-U = T_{Rolle}+T_{Masse}-mgh$
\begin{align*}
  T_{Masse} &= \frac{1}{2}m\dot{h}^2\\
  T_{Rolle} &= \frac{1}{2}\theta\dot{\phi}^2,
            &\phi = \phi(h) = 2\pi \frac{h}{2\pi R} = \frac{h}{R};
            &&\dot{\phi}= \frac{\dot{h}}{R}\\
  \Rightarrow
  L(h,\dot{h}) &= \frac{1}{2}\left( \frac{1}{2R^2}\theta + m \right)\dot{h}^2 + mgh
\end{align*}

\end{itemize}

\item\label{item:7} 

\end{enumerate}



%%% Local Variables:
%%% mode: latex
%%% TeX-master: "Script"
%%% End:
