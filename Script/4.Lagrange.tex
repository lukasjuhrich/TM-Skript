\section{Lagrange-Formalismus}
\subsection{Lagrange I}
\subsection{Lagrange II}
Betrachte folgend ein System mit $f$ Freiheitsgraden.
\begin{itemize}
\item $(q_1,\cdots,q_f)$ – „Generalisierte
  real.(?) Koordinaten“
\item ${q_L}$, $L=1,\cdots,f$ generalisierte Koordinate (?)
\item $(\dot{q_1}\cdots,\dot{q_f})$ „Verallgemeinerte
  Geschwindigkeiten“
\item Lagrange-Funktion: $L := T - U$\footnote{Anm.d.Skr.: im
    Englischen Wird die Potentielle Energie häufig mit $V$ bezeichnet,
    weswegen man oft (vor Allem auf Wikipedia) auf $L=T-V$ trifft.}
\item
  $L = L(\underbrace{q_1, \cdots, q_f}_f, \underbrace{\dot{q_1},
    \cdots, \dot{q_f}}_{f}, t) = L(q_L, \dot{q_L}, t) = L(q, \dot{q},
  t)$
% TODO markieren
\item Euler-Lagrange-Gleichungen:
  \begin{align*} \frac{d}{dt}\frac{\partial L}{\partial
      \dot{q_{\mathcal{L}}}} \frac{\partial L}{\partial
      q_{\mathcal{L}}} &= 0 & \mathcal{L} \in \left\{ 1,\cdots,f
    \right\}
  \end{align*}
\item Führt zu $f$ gekoppelten DGLs 2. Ordnung
\end{itemize}

\paragraph*{Beispiele:}
\subparagraph{Teilchen in 3D unter Einfluss eines Ortsabhängigen  Potentials}
\begin{itemize}
\item \emph{Keine Zwangsbedingungen} $\Rightarrow$
  $T = \frac{1}{2}m\dot\vec{r}$, $U=U \left( \vec{r} \right)$
\item $f=3$
\item Wir wählen \emph{kartesische Koordinaten}: $\left( q_1,q_2,q_3 \right) = (x,y,z)$
\item $\Rightarrow$
  $T = \frac{1}{2}m \left( \dot{x}^2 + \dot{y}^2 + \dot{z}^2 \right)$,
  $U = U(x,y,z)$ mit
  $\vec{r} = \vec{r}(q) = (x,y,z)^T$
\item $\Rightarrow$ für $\mathcal{L}\in \left\{ 1,2,3 \right\}$
  $L = L(x,y,z,\dot{x},\dot{y},\dot{z},t) = \frac{1}{2}m \left(
    \dot{x}^2 + \dot{y}^2 + \dot{z}^2 \right) - U(x,y,z)$
\item Euler-Lagrange-Gleichungen: Sei $\mathcal{L}=1$
% TODO fix alignment
\begin{align*}
  \Leftrightarrow
  \frac{d}{dt}\frac{\partial L}{\partial \dot{x}} \frac{\partial L}{\partial x} = 0\\
  \frac{\partial L}{\partial \dot{x}} = m\dot{x}
  \frac{\partial L}{\partial x}
&= - \frac{\partial U}{\partial x}\\
  \frac{d}{dt}m\dot{x} + \frac{\partial U}{\partial x}
&= 0
  \Leftrightarrow m \ddot{x} = - \frac{\partial U}{\partial x}
\end{align*}
\item Entsprechender Ansatz für $y,z$ führt insgesamt zu $m\ddot\vec{r}=-\nabla U$
\item Wie erwartet: Newton-Ansatz bestätigt!
\end{itemize}

\subparagraph{Mit Zwangsbedingungen: Mehre Beispiele}
\begin{enumerate}

\item\label{item:8} Rolle

\skizze{skizze}

\begin{itemize}
\item Hier: $f=1$, da (1D+1D)-Bewegung gekoppelt
\item Wähle Höhe $q_1=h$ (alternativ: Winkel $\varphi$) als
  generalisierte Koordinate
\item $L = L (h,\dot{h}) = T-U = T_{Rolle}+T_{Masse}-mgh$
\begin{align*}
  T_{Masse} &= \frac{1}{2}m\dot{h}^2\\
  T_{Rolle} &= \frac{1}{2}\theta\dot{\varphi}^2,
            &\varphi = \varphi(h) = 2\pi \frac{h}{2\pi R} = \frac{h}{R};
            &&\dot{\varphi}= \frac{\dot{h}}{R}\\
  \Rightarrow
  L(h,\dot{h}) &= \frac{1}{2}\left( \frac{1}{2R^2}\theta + m \right)\dot{h}^2 + mgh
\end{align*}

\end{itemize}

\item\label{item:7} Gedöns % TODO add description
\skizze{skizze}


$f=2$. z.B. $(q_1, q_2) = (x,y)$. Besser: $(q_1,q_2) = (\rho,\varphi)$ mit
\begin{align*}
\rho &= \sqrt{x^2+y^2} \\
x &= \rho \cos \varphi \\
y &= \rho \sin \varphi
&\Rightarrow L = L \left( \rho, \varphi, \dot{\rho} ,\dot{\varphi} \right)
\end{align*}

\item\label{item:9} Masse mit Loch auf Stange
\skizze{skizze}

Hier $f=2$ z.B. $(q_1,q_2) = (x,\varphi)$

% TODO box or sth
\begin{description}
\item[Zur Erinnerung]: In Lagrange I
\begin{align*}
  m_1 \ddot{\vec{r_1}} &= \vec{F_1} + \vec{Z_1} \\
  2 = f &= 3n - k = 6-4
\end{align*}
($f=6$ wenn nicht gekoppelt).

$\Rightarrow$ 4 Zwangsbedingungen:
\begin{align*}
  g_1 \left( \vec{r_1}, \vec{r_2} \right)
  &= z_1 = 0 \\
  g_2 \left( \vec{r_1}, \vec{r_2} \right)
  &= y_1 = 0 \\
  g_3 \left( \vec{r_1}, \vec{r_2} \right)
  &= y_2 = 0 \\
  g_4 \left( \vec{r_1}, \vec{r_2} \right)
  &= \underbrace{\left( \vec{r_1-r_2} \right)^2}_{\text{Abstand}}-l^2 = 0 \\
\end{align*}
\begin{align*}
  \Rightarrow \vec{z_1} = \sum\limits_{m=1}^4\lambda_m(t)
  \vec{\nabla}_ig_m(\vec{r_1}, \vec{r_2}) \quad i = 1,2
\end{align*}
\item[Lagrange II] $f=2$; $(q_1,q_2)=(x,\varphi)$
\begin{align*}
  \vec{r_1} &= \vec{r_1}(x,\varphi) = r_1(x)
  &&= \left( \begin{matrix}x\\0\\0 \end{matrix} \right)
            &&\text{für}\:m_1\\
  \vec{r_2} &= \vec{r_2}(x,\varphi)
  &&= \left( \begin{matrix}x + l\sin\varphi\\0\\-l\cos\varphi \end{matrix} \right)
            &&\text{für Pendelmasse}\\
\end{align*}
\begin{align*}
  T &= \frac{1}{2}m_1\vec{r}_1^2 + \frac{1}{2}m_2\vec{r}_2^2
  &= T \left( x,\varphi,\dot{x},\dot{\varphi} \right)\\
    &\dot\vec{r}_1 = \left(\begin{matrix}\dot{x}\\0\\0\end{matrix}\right)
  \qquad\dot\vec{r}_2 = \left(\begin{matrix}\dot{x}
      + l\dot{\varphi}\cos\varphi\\0\\l\dot{\varphi}\sin\varphi\end{matrix}\right)\\
  T &= \frac{1}{2}m_1\dot{x}^2
      + \frac{1}{2}m_2 \left( (\dot{x}+l\dot{\varphi}\cos\varphi)^2 + l^2\varphi^2\sin^2\varphi \right)
  &= T \left( x,\varphi,\dot{x},\dot{\varphi} \right)\\
    &= \frac{1}{2}(m_1+m_2)\dot{x}^2 + m_2l\dot{x}\dot{\varphi}\cos\varphi
      + \frac{1}{2}ml^2\dot{\varphi}^2\\
  &U(\vec{r}_1,\vec{r}_2) = mgz_2 = -mgl\cos\varphi = U(x,\varphi)
  &L = L(x,\varphi,\dot{x},\dot{\varphi})\\
\end{align*}

\item[Bewegungsgleichung:]
  $\frac{d}{dt}\frac{\partial L}{\partial \dot{x}} - \frac{\partial
    L}{\partial x} = 0$. 
  Hier:
\begin{align*}
  \frac{\partial L}{\partial \dot{x}}
  &= (m_1+m_2)\dot{x}+m_2l\dot{\varphi}\cos\varphi\\
  \frac{d}{dt}\frac{\partial L}{\partial \dot{x}}
  &= (m_1+m_2)\ddot{x} + m_2l\ddot{\varphi}\cos\varphi\\
\end{align*}
entsprechend
\begin{equation*}
\frac{d}{dt}\frac{\partial L}{\partial \dot{\varphi}}-\frac{\partial L}{\partial \varphi} = 0
\qquad \ddot{\varphi} = 0
\end{equation*}
mit $\omega^2=\frac{m_1+m_2}{m_1}\frac{l}{g}$

$\Rightarrow$ für kleinere Auslenkungen $\varphi<<1$: $\ddot{\varphi}+\omega^2\varphi=0$

\item[Bemerkungen]
\begin{itemize}
\item Lagrange II sehr nützlich. Beweis: klar.
\item Euler-Lagrange-Gleichungen invariant unter
  Koordinatentransformation
  \begin{align*}
    \vec{r}_i(q_1,\cdots,q_p,t)
    &\leftrightarrow \vec{r}_i(Q_1,\cdots,Q_p,t)\\
    \frac{d}{dt}\frac{\partial L}{\partial \dot{q}_k}-\frac{\partial L}{\partial q_L} = 0
    &\Leftrightarrow \frac{d}{dt} \frac{\partial L^{\prime}}{\partial \dot{Q}_{\beta}}
      - \frac{\partial L^{\prime}}{\partial Q_{\beta}}
  \end{align*}
  Wobei
  $L(q,\dot{q},t) f L^{\prime}(Q,\dot{Q},t- =
  L^{\prime}(Q_{\beta}(q_1,\cdots,q_p),\dot{Q}_{\beta}(q_1,\cdots,q_p,
  \dot{q}_1,\cdots,\dot{q}_p),t)$

  Transformation
  $\left\{ q_L \right\} \rightarrow \left\{ Q_{\beta} \right\}$;
  $Q_{\beta}=Q_{\beta}(q_1,\cdots,q_p)$, $\beta = 1,\cdots,f$

  Im Gegensatz zu Newton!  Es gilt
  $m\ddot{x} = - \frac{\partial U}{\partial x}$ bei Übergang zu
  Kugelkoordinaten $(x,y,z)\rightarrow(r,\theta,\varphi)$ es gilt
  \emph{nicht}
  $m\ddot{\varphi} \theta \not= - \frac{\partial U}{\partial \varphi}$ Widersp.
\item Bisher $\vec{F}_i = \vec{\nabla}_iU(\vec{r}_1,\cdot,\vec{r}_N)$
  analog für verallgemeinerte Kräfte $\vec{K}_i$, für die gilt:
  \begin{align*}
    \vec{K}_i &= -\vec{\nabla}_{\vec{r}_i}\tilde{U}(\vec{r},\dot\vec{r})
                + \frac{d}{dt}\vec{\nabla}_{\dot\vec{r}_i} \tilde{U}(\vec{r},\dot\vec{r})
  \end{align*}
  es folgt wieder
  $m\ddot\vec{r}_i = \vec{K}_{i} \Leftrightarrow
  \frac{d}{dt}\frac{\partial L}{\partial q_L} = 0, \qquad L = T -
  \tilde{U}(\vec{r},\dot\vec{r})$

  \begin{description}
  \item[Beispiel] Loretzkraft:
    \begin{align*}
      \tilde{U}(\vec{r},\vec{\dot{r}})
      = e \underbrace{\phi(\vec{r},t)}_{\mathclap{\text{Skalares Pot.}}} - e
      \frac{\dot\vec{r}}{c} \cdot
      \underbrace{\vec{A}(\vec{r},t)}_{\mathclap{\text{Vektorpot.}}} 
    \end{align*}
  \end{description}

\end{itemize}

\end{description}

\end{enumerate}



%%% Local Variables:
%%% mode: latex
%%% TeX-master: "Script"
%%% End:
