\section{Lagrange-Formalismus}

\subsection{Lagrange I}

% TODO Vervollständigenn

\subsection{Lagrange II}

Betrachte folgend ein System mit $f$ Freiheitsgraden.
\begin{itemize}
\item $(q_1,\cdots,q_f)$ – „Generalisierte
  real.(?) Koordinaten“
\item ${q_L}$, $L=1,\cdots,f$ generalisierte Koordinate (?)
\item $(\dot{q_1}\cdots,\dot{q_f})$ „Verallgemeinerte
  Geschwindigkeiten“
\item Lagrange-Funktion: $L := T - U$\footnote{Anm.d.Skr.: im
    Englischen Wird die Potentielle Energie häufig mit $V$ bezeichnet,
    weswegen man oft (vor Allem auf Wikipedia) auf $L=T-V$ trifft.}
\item
  $L = L(\underbrace{q_1, \cdots, q_f}_f, \underbrace{\dot{q_1},
    \cdots, \dot{q_f}}_{f}, t) = L(q_L, \dot{q_L}, t) = L(q, \dot{q},
  t)$
% TODO markieren
\item Euler-Lagrange-Gleichungen:
  \begin{align*} \diff{}{t}\pdiff{L}{\dot{q_{\mathcal{L}}}}
    \pdiff{L}{q_{\mathcal{L}}} &= 0 & \mathcal{L} \in
    \left\{ 1,\cdots,f \right\}
  \end{align*}
\item Führt zu $f$ gekoppelten DGLs 2. Ordnung
\end{itemize}

\paragraph*{Beispiele:}
\subparagraph{Teilchen in 3D unter Einfluss eines Ortsabhängigen  Potentials}
\begin{itemize}
\item \emph{Keine Zwangsbedingungen} $\Rightarrow$
  $T = \frac{1}{2}m\dot\vec{r}$, $U=U \left( \vec{r} \right)$
\item $f=3$
\item Wir wählen \emph{kartesische Koordinaten}: $\left( q_1,q_2,q_3 \right) = (x,y,z)$
\item $\Rightarrow$
  $T = \frac{1}{2}m \left( \dot{x}^2 + \dot{y}^2 + \dot{z}^2 \right)$,
  $U = U(x,y,z)$ mit
  $\vec{r} = \vec{r}(q) = (x,y,z)^T$
\item $\Rightarrow$ für $\mathcal{L}\in \left\{ 1,2,3 \right\}$
  $L = L(x,y,z,\dot{x},\dot{y},\dot{z},t) = \frac{1}{2}m \left(
    \dot{x}^2 + \dot{y}^2 + \dot{z}^2 \right) - U(x,y,z)$
\item Euler-Lagrange-Gleichungen: Sei $\mathcal{L}=1$
% TODO fix alignment
\begin{align*}
  \Leftrightarrow
  \diff{}{t}\pdiff{L}{\dot{x}} \pdiff{L}{x} = 0\\
  \pdiff{L}{\dot{x}} = m\dot{x}
  \pdiff{L}{x}
&= - \pdiff{U}{x}\\
  \diff{}{t}m\dot{x} + \pdiff{U}{x}
&= 0
  \Leftrightarrow m \ddot{x} = - \pdiff{U}{x}
\end{align*}
\item Entsprechender Ansatz für $y,z$ führt insgesamt zu $m\ddot\vec{r}=-\nabla U$
\item Wie erwartet: Newton-Ansatz bestätigt!
\end{itemize}

\subparagraph{Mit Zwangsbedingungen: Mehre Beispiele}
\begin{enumerate}

\item\label{item:8} Rolle

\skizze{skizze}

\begin{itemize}
\item Hier: $f=1$, da (1D+1D)-Bewegung gekoppelt
\item Wähle Höhe $q_1=h$ (alternativ: Winkel $\varphi$) als
  generalisierte Koordinate
\item $L = L (h,\dot{h}) = T-U = T_{Rolle}+T_{Masse}-mgh$
\begin{align*}
  T_{Masse} &= \frac{1}{2}m\dot{h}^2\\
  T_{Rolle} &= \frac{1}{2}\theta\dot{\varphi}^2,
            &\varphi = \varphi(h) = 2\pi \frac{h}{2\pi R} = \frac{h}{R};
            &&\dot{\varphi}= \frac{\dot{h}}{R}\\
  \Rightarrow
  L(h,\dot{h}) &= \frac{1}{2}\left( \frac{1}{2R^2}\theta + m \right)\dot{h}^2 + mgh
\end{align*}

\end{itemize}

\item\label{item:7} Gedöns % TODO add description
\skizze{skizze}


$f=2$. z.B. $(q_1, q_2) = (x,y)$. Besser: $(q_1,q_2) = (\rho,\varphi)$ mit
\begin{align*}
\rho &= \sqrt{x^2+y^2} \\
x &= \rho \cos \varphi \\
y &= \rho \sin \varphi
&\Rightarrow L = L \left( \rho, \varphi, \dot{\rho} ,\dot{\varphi} \right)
\end{align*}

\item\label{item:9} Masse mit Loch auf Stange
\skizze{skizze}

Hier $f=2$ z.B. $(q_1,q_2) = (x,\varphi)$

% TODO box or sth
\begin{description}
\item[Zur Erinnerung]: In Lagrange I
\begin{align*}
  m_1 \ddot{\vec{r_1}} &= \vec{F_1} + \vec{Z_1} \\
  2 = f &= 3n - k = 6-4
\end{align*}
($f=6$ wenn nicht gekoppelt).

$\Rightarrow$ 4 Zwangsbedingungen:
\begin{align*}
  g_1 \left( \vec{r_1}, \vec{r_2} \right)
  &= z_1 = 0 \\
  g_2 \left( \vec{r_1}, \vec{r_2} \right)
  &= y_1 = 0 \\
  g_3 \left( \vec{r_1}, \vec{r_2} \right)
  &= y_2 = 0 \\
  g_4 \left( \vec{r_1}, \vec{r_2} \right)
  &= \underbrace{\left( \vec{r_1-r_2} \right)^2}_{\text{Abstand}}-l^2 = 0 \\
\end{align*}
\begin{align*}
  \Rightarrow \vec{z_1} = \sum\limits_{m=1}^4\lambda_m(t)
  \vec{\nabla}_ig_m(\vec{r_1}, \vec{r_2}) \quad i = 1,2
\end{align*}
\item[Lagrange II] $f=2$; $(q_1,q_2)=(x,\varphi)$
\begin{align*}
  \vec{r_1} &= \vec{r_1}(x,\varphi) = r_1(x)
  &&= \left( \begin{matrix}x\\0\\0 \end{matrix} \right)
            &&\text{für}\:m_1\\
  \vec{r_2} &= \vec{r_2}(x,\varphi)
  &&= \left( \begin{matrix}x + l\sin\varphi\\0\\-l\cos\varphi \end{matrix} \right)
            &&\text{für Pendelmasse}\\
\end{align*}
\begin{align*}
  T &= \frac{1}{2}m_1\vec{r}_1^2 + \frac{1}{2}m_2\vec{r}_2^2
  &= T \left( x,\varphi,\dot{x},\dot{\varphi} \right)\\
    &\dot\vec{r}_1 = \left(\begin{matrix}\dot{x}\\0\\0\end{matrix}\right)
  \qquad\dot\vec{r}_2 = \left(\begin{matrix}\dot{x}
      + l\dot{\varphi}\cos\varphi\\0\\l\dot{\varphi}\sin\varphi\end{matrix}\right)\\
  T &= \frac{1}{2}m_1\dot{x}^2
      + \frac{1}{2}m_2 \left( (\dot{x}+l\dot{\varphi}\cos\varphi)^2 + l^2\varphi^2\sin^2\varphi \right)
  &= T \left( x,\varphi,\dot{x},\dot{\varphi} \right)\\
    &= \frac{1}{2}(m_1+m_2)\dot{x}^2 + m_2l\dot{x}\dot{\varphi}\cos\varphi
      + \frac{1}{2}ml^2\dot{\varphi}^2\\
  &U(\vec{r}_1,\vec{r}_2) = mgz_2 = -mgl\cos\varphi = U(x,\varphi)
  &L = L(x,\varphi,\dot{x},\dot{\varphi})\\
\end{align*}

\item[Bewegungsgleichung:]
  $\diff{}{t}\pdiff{L}{\dot{x}} - \pdiff{
    L}{x} = 0$.
  Hier:
\begin{align*}
  \pdiff{L}{\dot{x}}
  &= (m_1+m_2)\dot{x}+m_2l\dot{\varphi}\cos\varphi\\
  \diff{}{t}\pdiff{L}{\dot{x}}
  &= (m_1+m_2)\ddot{x} + m_2l\ddot{\varphi}\cos\varphi\\
\end{align*}
entsprechend
\begin{equation*}
\diff{}{t}\pdiff{L}{\dot{\varphi}}-\pdiff{L}{\varphi} = 0
\qquad \ddot{\varphi} = 0
\end{equation*}
mit $\omega^2=\frac{m_1+m_2}{m_1}\frac{l}{g}$

$\Rightarrow$ für kleinere Auslenkungen $\varphi<<1$: $\ddot{\varphi}+\omega^2\varphi=0$

\item[Bemerkungen]
\begin{enumerate}[i)]

\item Lagrange II sehr nützlich. Beweis: klar.

\item Euler-Lagrange-Gleichungen invariant unter
  Koordinatentransformation
  \begin{align*}
    \vec{r}_i(q_1,\cdots,q_p,t)
    &\leftrightarrow \vec{r}_i(Q_1,\cdots,Q_p,t)\\
    \diff{}{t}\pdiff{L}{\dot{q}_k}-\pdiff{L}{q_L} = 0
    &\Leftrightarrow \diff{}{t} \pdiff{L^{\prime}}{\dot{Q}_{\beta}}
      - \pdiff{L^{\prime}}{Q_{\beta}}
  \end{align*}
  Wobei
  $L(q,\dot{q},t) f L^{\prime}(Q,\dot{Q},t- =
  L^{\prime}(Q_{\beta}(q_1,\cdots,q_p),\dot{Q}_{\beta}(q_1,\cdots,q_p,
  \dot{q}_1,\cdots,\dot{q}_p),t)$

  Transformation
  $\left\{ q_L \right\} \rightarrow \left\{ Q_{\beta} \right\}$;
  $Q_{\beta}=Q_{\beta}(q_1,\cdots,q_p)$, $\beta = 1,\cdots,f$

  Im Gegensatz zu Newton!  Es gilt
  $m\ddot{x} = - \pdiff{U}{x}$ bei Übergang zu
  Kugelkoordinaten $(x,y,z)\rightarrow(r,\theta,\varphi)$ es gilt
  \emph{nicht}
  $m\ddot{\varphi} \theta \not= - \pdiff{U}{\varphi}$ Widersp.
\item Bisher $\vec{F}_i = \vec{\nabla}_iU(\vec{r}_1,\cdot,\vec{r}_N)$
  analog für verallgemeinerte Kräfte $\vec{K}_i$, für die gilt:
  \begin{align*}
    \vec{K}_i &= -\vec{\nabla}_{\vec{r}_i}\tilde{U}(\vec{r},\dot\vec{r})
                + \diff{}{t}\vec{\nabla}_{\dot\vec{r}_i} \tilde{U}(\vec{r},\dot\vec{r})
  \end{align*}
  es folgt wieder
  $m\ddot\vec{r}_i = \vec{K}_{i} \Leftrightarrow
  \diff{}{t}\pdiff{L}{q_L} = 0, \qquad L = T -
  \tilde{U}(\vec{r},\dot\vec{r})$

  \begin{description}
  \item[Beispiel] Loretzkraft:
    \begin{align*}
      \tilde{U}(\textbf{r},\dot{\textbf{r}})
      = e \underbrace{\phi(\textbf{r},t)}_{\mathclap{\text{Skalares Pot.}}} - e
      \frac{\dot{\textbf{r}}}{c} \cdot
      \underbrace{\textbf{A}(\textbf{r},t)}_{\mathclap{\text{Vektorpot.}}} \\
      \Rightarrow \textbf{K} = e \left( E \left( \textbf{r}, t\right)
      + \frac{\dot{\textbf{r}}}{c} \times \textbf{B} \left( \dot{\textbf{r}}, t \right)
      \right)
    \end{align*}
    mit $\textbf{E} = - \nabla \Phi - \frac{1}{C} \pdiff{}{t} \textbf{A},
    \quad \textbf{B} = \nabla \times \textbf{A}$
  \end{description}

  % TODO Box
  Die Lagrange-Funktion eines geladenen Teilchens ($e$) im
  elektromagnetischen Feld Lautet:

  \begin{equation*}
    L(\mathbf{r}, \dot{\mathbf{r}}, t) = \frac{1}{2} m \dot{\mathbf{r}}^2
    - e \Phi(\mathbf{r}, t) + e \frac{\dot{\mathbf{r}}}{c} \mathbf{A}(\mathbf{r}, t)
  \end{equation*}

\item Man nennt $P_{\alpha} = \pdiff{L}{\dot{q_{\alpha}}}$ den
  „generalisierten Impuls“
  ($ l = \frac{1}{2} m \dot{x}^2, p = \pdiff{L}{\dot{x}} = m\dot{x}$)

\item Eine generalisierte Koordinate $q_{\alpha}$ heißt
  \emph{zyklisch}, falls die Lagrange-Funktion unabhängig von
  $q_{\alpha}$ ist: $\pdiff{L}{q_{\alpha}} = 0$

  \begin{description}
  \item[Beispiel] $q_1 = x$ (im vierten Beispiel)

    Es gilt $ \diff{}{t} \pdiff{L}{\dot{q_{\alpha}}} = 0 \Rightarrow $
    Erhaltungssatz ($P_{\alpha} = \mathsf{const.}$) (Siehe
    Noether-Theorem)
  \end{description}

Problem mit Zylindersymmetrie:
  $V(\mathbf{r}) = V(\rho, \varphi, z)
  \stackrel{\mathclap{\mathsf{Sym.}}}{=} V(\rho,z)$

\begin{align*}
  x = \rho \cos \varphi \quad
  L = \frac{1}{2} m \left( x^2+y^2+z^2 \right) - V \left( \rho, z \right) \\
  y = \sin \varphi \\
  L \left( \rho, (\phi), z, \dot{\rho}, \dot{\varphi}, \dot{z} \right)
  = \frac{1}{2} m \left( \dot{\rho}^2 + \rho^2 \dot{\varphi}^2 + \dot{z}^2 \right)
  - V(\rho,z)
\end{align*},

$\varphi$ zyklisch: $\pdiff{L}{\varphi} = 0$

\begin{description}
\item[Generalisierter Impuls]
  $ \underbrace{\mathsf{const.} =
    p_{\varphi}}_{\mathclap{\mathsf{Erhaltungsgr.}}}  =
  \pdiff{L}{\dot{\varphi}} = (\mathbf{L})_3 $ (z-Komponente des Drehimpulses)

\end{description}

\item Diskussion von Kreiselbewegung „Schwerer Kreisel“:

\begin{align*}
  \mathbf{\Omega} \rightarrow \mathbf{\Omega} \underbrace{ \left(
  \varphi, \dot{\varphi}, \psi, \dot{\psi}, \theta, \dot{\theta}
  \right)}_{\mathsf{Eulerwinkel}}\\
  \Rightarrow L = L \left(
  \varphi, \psi, \theta, \dot{\psi}, \dot{\varphi}, \dot{\theta}
  \right) = \frac{1}{2} \mathbf{\Omega}^{\top} \underline{G} \mathbf{\Omega}
  - U(\theta);
  \quad \overbrace{U(\theta) = mga\cos\theta}^{E_{pot}\:\mathsf{Kreisel}}
\end{align*}


\end{enumerate}

\end{description}

% TODO refactor where this should end.
\end{enumerate}


\subsection{Wirkung und Hamiltonsches Prinzip}

\begin{description}
\item[Beispiel] Vorlesung, Studenten im Hörsaal. Student $A$ soll
  Professor $B$ ein Glas Wasser bringen, und dies schnellstmöglich.

\skizze{}

\item[Frage] Was ist die \emph{schnellste} Verbindung $A \rightarrow B$?

  Zeit
  \begin{align*}
    T_{A \rightarrow B} = T(x) = T_1+T_{2} = \frac{l_1}{v_1} +
    \frac{l_2}{v_2}
  \end{align*}
  Länge
  \begin{align*}
    l_1(x) &= \sqrt{x^2+ (a-d)^2}
    & l_2(x) &= \sqrt{(b-x)^2 + d^2}\\
    l'_1(x) &= \frac{x}{l_{1}(x)}
    & l'_2(x) &= \frac{b-x}{l_{2}(x)}
  \end{align*}

\skizze{}

Aus $\pdiff{T}{x} \big|_{x=\bar{x}} = 0$ folgt:

\begin{align*}
  T'(x) &= \frac{v_2 l'_1(x) + v_1 l'_2(x)}{v_1v_2}\\
  T'(x) &= 0 \Leftrightarrow v_2 \frac{x}{l_1(x)}
          = v_1 \frac{b-x}{l_2(x)} = 0\\
  T'(x) &= 0 \Leftrightarrow \boxed{ \frac{\sin{\alpha}}{\sin{\alpha_2}}
          = \frac{v_1}{v_2} }
          \quad = \frac{n_2}{n_1} \left( v_i = \frac{c}{n_i}  \right)
          \textnormal{Brechungsgesetz}
\end{align*}

\end{description}

\paragraph{Ähnliche Fragen der Mechanik:}
\begin{description}

\item[Frage] Welche Kurve $z(t)$ führt zur schnellsten Dynamik von
  $A \rightarrow B$?

  \skizze{}

  Vorgehensweise wie zuvor: Berechne die Laufzeit $T = T[z]$ -- $z(x)$
  ist \emph{Funktion}

  \skizze{}

  $\rightarrow$ Suche \emph{Minimum} des \emph{Funktionals} $T[z]$
  ($T(z(x))$) im Raum der Kurven $z(x)$.

  Bei Diskretisierung $z_i = z(x_i)$:
  $T[z] \rightarrow T (z_1, \dots, z_n)$ gewöhnliche Funktion.

  Bestimme jene
  $\lbrace z_i \:|\: i\in \lbrace 1, \dots, N \rbrace \rbrace$, für
  die
  $$dF = \sum\limits_i \pdiff{T}{z_i} dz_i = 0 \Leftrightarrow
  \pdiff{T}{z_i} = 0 $$ ($i \in \lbrace 1, \dots, N \rbrace$)

\item[Hier:] Funktional $T[z]$ mit $z(x)$ Funktion, bilde
  $\sum\limits_i \rightarrow \int dx$,
  $\sum\limits_{i=1}^N \pdiff{T}{z_i} \rightarrow \int dx \frac{\delta
    T}{\delta z(x)} \delta z(x)$ % TODO Pfeile

  Sodass das Minimum derer
  $0 \stackrel{!}{=} \delta T = \int dx \left( \frac{\delta T}{\delta
      z(x)} \delta z(x) \right) \Leftrightarrow$

% TODO *hier* weitermachen :)

\end{description}


%%% Local Variables:
%%% mode: latex
%%% TeX-master: "Script"
%%% End:
