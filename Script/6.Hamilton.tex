\section{Hamiltonsche Mechanik}

$$
q(t) \rightarrow (q(t), p(t))
q\dot{q} 
$$

\subsection{Poisson-Klammer}
\begin{itemize}
\item Zwei Phasenraumfunktionen $f(q,p), g(q_\alpha,p_\alpha)$ wird
  eine neue Phasenraumfunktion
  $\left\{ f,g \right\}(q_\alpha,p_\alpha) := \sum\limits_\alpha
  \left( \frac{\partial f}{\partial q_\alpha}\frac{\partial
      q}{\partial p_\alpha} - \frac{\partial f}{\partial
      p_\alpha}\frac{\partial q}{\partial q_\alpha} \right)$
  zugeordnet

\item Diese hat zuvorkommende Eigenschaften:
\begin{itemize}
\item Zeitabhängigkeit: $A(q,p,t)$ entlang einer Trajektorie
  bzw. Lösung der Hamiltonschen Bewegungsgleichung $A(q(t),p(t),t)$
$$\frac{d}{dt} A = \left\{ A,H \right\} + \frac{\partial}{\partial t}A$$
\item Insbesondere: $A(q,p)$ Erhaltungsgröße
  $\Rightarrow \left\{ A,H \right\}=0$.  Sehr praktisch, um zu prüfen,
  ob etwas eine Erhaltungsgröße ist.

\item ($H(q,p)$ Erhaltungsgröße, da
  $\forall f:\left\{ f,f \right\}=0$)
\end{itemize}

\item Eigenschaften der Poisson-Klammer:
  \begin{enumerate}
  \item\label{item:1} $\left\{ f,g \right\}=-\left\{ g,f \right\}$
  \item\label{item:2}
    $\left\{ f,g+h \right\}=\left\{ f,g \right\}+\left\{ f,h \right\}$
  \item\label{item:3}
    $\left\{ f,gh \right\}=g\left\{ f,h \right\} + \left\{ f,g
    \right\}h$
  \item\label{item:4}
    $\left\{ f,\left\{ g,h \right\} \right\} + \left\{ h ,\left\{f,g
      \right\} \right\} + \left\{ g\left\{ h,f \right\} \right\} = 0$
    Bemerkung.: Ist $f$ und $g$ Erhaltungsgröße
    $\Rightarrow \left\{ f,g \right\}$ Erhaltungsgröße
  \item\label{item:5} Elementare Poisson-Klammern:
    $\left\{ q,p \right\} = 1$,
    $\left\{ q,q \right\}=\left\{ p,p \right\}=0$.

  \item\label{item:6} Anwendung:
    $\left\{ q^2,p \right\}q = q \left\{ q,p \right\} + \left\{ q,p
    \right\}q = 2q$
  \end{enumerate}

\end{itemize}


\subsection{Kanonische Transformationen}
\begin{itemize}
\item Bislang (Lagrange):
  $\left\{ q_{\alpha}\right\}\rightarrow \left\{
    Q_{\alpha}(q_1,\cdots,q_f,t \right\}$
  \emph{Koordinaten}transformationen
\item Jetzt (Hamilton):
  $\left\{ q_{\alpha}, p_{\alpha} \right\} \rightarrow \left\{
    Q_{\alpha}(q,p,t), P_{\alpha}(q,p,t) \right\}$
  \emph{Phasenraum}transformationen
\item Falls
  $\dot{q_{\alpha}} = \frac{\partial H}{\partial p_{\alpha}}
  \dot{p_{\alpha}} = -\frac{\partial H}{\partial q_{\alpha}}
  \Rightarrow \dot{Q_{\alpha}} = \frac{\partial K}{\partial
    P_{\alpha}} \dot{P_{\alpha}} = -\frac{\partial K}{\partial
    Q_{\alpha}}$.

  Ein solches $K$ existiert $\Leftrightarrow$ Transformation
  $(q,p) \rightarrow(Q,P)$ Kanonisch
\item Betrachte Phasenraumvolumina:
  \skizze{skizze}

\begin{align*}
F_S = \int d Q \int dP = \int dq \int dp \begin{vmatrix}
\frac{\partial Q}{\partial q} & \frac{\partial P}{\partial p} \\
\frac{\partial Q}{\partial p} & \frac{\partial P}{\partial q} \\
\end{vmatrix}
\end{align*}

\item Phasenraumvolumen bleibt erhalten:
  $F_R = F_S \Leftrightarrow \left| \cdot \right| = 1 = \cdots =
  \left\{ Q,P \right\}$

\item Definition: Eine Phasenraumtransformation
  $T: (q_{\alpha}, p_{\alpha} \rightarrow (Q_{\alpha}(q,p,t),
  P_{\alpha}(q,p,t))$
  heißt \emph{kanonisch} $\Leftrightarrow$ das Phasenraumvolumen
  bleibt erhalten ($\Leftrightarrow V_R=V_S$).

\item Es gilt: $-$ kanonisch
  $\Leftrightarrow \left\{ Q_{\alpha},P_{\beta} \right\} =
  \delta_{\alpha\beta} $
  $ \left( f-1:\quad \left\{ Q,P \right\}_{(q,p)} = 1 \right) $
  $\left\{ Q_{\alpha}, Q_{\beta} \right\} = 0$

$\left\{ P_{\alpha}, P_{\beta} \right\} = 0$

\item Alles dreis: für $\alpha,\beta = 1,2,\cdots,f$
  % todo format a bit nicer.
  $\rightarrow$ (Hier kommt irgendwas hin, keine Ahnung was)

\item Bemerkung zur Poisson-Klammer: es gilt auch
  $\left\{ Q,P \right\}_{(Q,P)} = 1$; dahinter steckt die
  \emph{Invarianz der Poisson-Klammer} unter kanonischen
  Transformationen
\begin{align*}
T: d \left\{ f,g \right\}_{(q,p)} = \left\{ f,g \right\}_{(Q,P)}
&&\left(f(q,p) \rightarrow f(q(Q,P), p(Q,P)) \right) 
\end{align*}
\end{itemize}


\subsection{(Form-)Invarianz der Hamiltonschen Bewegungsgleichungen
  unter kanonischen Transformationen}
\begin{itemize}
\item Ausgangspunkt:
  $\dot{q_{\alpha}} = \frac{\partial H(q,p)}{\partial
    p_{\alpha}};\quad \dot{p_{\alpha}} = - \frac{\partial H}{\partial
    q_{\alpha}}$
\item Betrachte die kanonische Transformation
  \begin{align*}
    T:\quad (q,p) \mapsto (Q,P) = \left(
    Q_{\alpha}(q_1,\cdots,q_f,p_1,\cdots,p_f ),
    P_{\alpha}(q_1,\cdots,q_f,p_1,\cdots,p_f) \right)
  \end{align*}
\item Für Zeitabhängigkeit der $\left( Q_{\alpha}, P_{\alpha} \right)$
  gilt
  \begin{align*}
    \dot{Q_{\alpha}}
    &= \left\{ Q_{\alpha}, H \right\}_{(q,p)}
      \stackrel{\mathclap{\text{kanT}}}{=} \left\{ Q_{\alpha}, H \right\}_{(Q,P)}\\
    &= \frac{\partial Q_{\alpha}}{\partial Q_{\alpha}}\frac{\partial H}{\partial P_{\alpha}}
      - \frac{\partial Q_{\alpha}}{\partial P_{\alpha}}\frac{\partial H}{\partial Q_{\alpha}}
      = \frac{\partial H(q,p)}{\partial P_{\alpha}}\\
    &= \frac{\partial K(Q,P)}{\partial P_{\alpha}}
  \end{align*}
\item Mit $K(Q,P) = H(q(Q,P),p(Q,P))$ ist genauso:
  \begin{align*}
    \dot{P_{\alpha}} = \left\{ P_{\alpha}, H \right\}_{(q,p)} = \left\{
    P_{\alpha},H \right\}_{(Q,P)} = - \frac{\partial K(Q,P)}{\partial
    Q_{\alpha}}
\end{align*}

\item Wie erwartet und erwünscht: Die klassischen Bewegungsgleichungen
  greifen!
\end{itemize}

\subsection{Erzeugende von kanonischen Transformationen}
\begin{itemize}
\item Ausgangspunkt: Hamiltonsches Prinzip
  $\delta S \stackrel{!}{=} 0$
\item
  $S = \int\limits_{t_1}^{t_2} dt \left\{ \dot{q} p - H(q,p) \right\}
  = \int\limits_{t_1}^{t_2}dt \left( \dot{Q}P - K(Q,P) \right) $
\item Linke Seite
  $pdq - Hdt = \underbrace{PdQ - K(Q,P) dt}_{\mathclap{\text{Rechte Seite}}} +
  \underbrace{dF}_{\mathclap{\text{Freiheit}}}$
\item Bei Variation liefern Beiträge des Randes keinen Beitrag!
\item Hier ist
  $F = F ( q,p,\underbrace{Q}_{\mathclap{Q(q,p),P(q,p)}},P,t )
  \stackrel{?}{=} F(q,Q,t)$
\item Fasse $F = F_1(q,Q,t)$ als Funktion dar alten und neuen
  Koordinaten auf:
  \begin{align*}
\label{eq:2}
dF_1 &= &\frac{\partial F_1}{\partial q}dq + \frac{ \partial F_1}{\partial Q}dQ + \frac{\partial F_1}{\partial t} dt\\
\Rightarrow pdq - Fdt &= PdQ - Kdt + &\frac{\partial F_1}{\partial q}dq + \frac{ \partial F_1}{\partial Q}dQ + \frac{\partial F_1}{\partial t} dt
\end{align*}

\begin{align*}
%\label{eq:1}
p &= \frac{\partial F_1}{\partial q}\\
P &= - \frac{\partial F_1}{\partial Q}\\
K &= H + \frac{\partial F_1}{\partial t}
\end{align*}
\item Jede Funktion $F_1(q,Q,t)$ erzeugt durch (*) (TODO label) eine
  kanonische Transformation.
\item Entsprechend lassen sich kanonische Transformationen erzeugen
  durch Erzeugende vom Typ
  \begin{align*}
F_2 = F_2(q,P,t) \Rightarrow &&p = -\frac{\partial F_2}{\partial q}, Q = \frac{\partial F_2}{\partial P}, k = H + \frac{\partial F_2}{\partial t}\\
F_3 = F_3(q,P,t) \Rightarrow &&p = -\frac{\partial F_3}{\partial q}, Q = \frac{\partial F_3}{\partial P}, k = H + \frac{\partial F_3}{\partial t}\\
F_4 = F_4(q,P,t) \Rightarrow &&p = -\frac{\partial F_4}{\partial q}, Q = \frac{\partial F_4}{\partial P}, k = H + \frac{\partial F_4}{\partial t}
\end{align*}

\end{itemize}


%%% Local Variables:
%%% mode: latex
%%% TeX-master: "Script"
%%% End:
